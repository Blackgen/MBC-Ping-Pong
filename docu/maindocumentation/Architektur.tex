\chapter{Architektur}

\section{Arbeitsablauf zur Bearbeitung eines Issue}
Um eine erfolgreiche Zusammenarbeit zu gewährleisten, sind allgemein gültige Regeln nötig. Insbesondere wird festgelegt, wie die einzelnen Arbeitsschritte ablaufen sollten, um ein Issue zu bearbeiten. Zudem werden weiterhin die Zuständigkeiten geregelt. 

\subsection{Verwaltung der zu bearbeitenden Issues}
Die zu bearbeitenden Issues werden auf GitHub unter Issues (\href{https://github.com/Transport-Protocol/MBC-Ping-Pong/issues}{https://github.com/Transport-Protocol/MBC-Ping-Pong/issues}) gepflegt. Um den Verlauf eines Issues darzustellen wird das Kanbanboard von GitHub (\href{https://github.com/Transport-Protocol/MBC-Ping-Pong/projects}{https://github.com/Transport-Protocol/MBC-Ping-Pong/projects}) genutzt.

\subsection{Erstellen der zu bearbeitenden Issues}
Prinzipiell kann und darf jedes Projektmitglied zu jeder Zeit Issues erstellen. Gerade bei Bugs ist dies ein gewünschtes vorgehen. In der Regel sollten dies jedoch aus Gruppensitzungen hervorgehen und durch den Architekten ausformuliert werden.\newline
Ein Issue besteht aus drei Absätzen:
\begin{itemize}
	\item \textbf{Beschreibung} \newline
	In der Beschreibung wird allgemein auf den Kontext des Issues eingegangen.
	\item \textbf{Anforderung} \newline
	In Anforderung wird die Zielvision dargestellt.
	\item \textbf{Abnahmekriterien} \newline
	In Abnahmekriterien werden alle Punkte aufgeführt, die notwendig sind, um das Issue als erfolgreich bearbeitet anzusehen.
\end{itemize}

\subsection{Das Kanbanboard}
Das Kanbanboard ist in fünf Abschnitte eingeteilt: 
\begin{itemize}
	\item \textbf{Selected for Development} \newline
	Diese Spalte enthält alle Issues, die der Architekt zur Bearbeitung in nächster Zeit ausgewählt hat. Hier enthaltene Issues sind entweder durch den Architekten einem bestimmten Teammitglied zugeordnet. Diese sollten dann auch vorrangig bearbeitet werden. Oder (dies sollte der Normalfall sein) sie sind niemandem zugeordnet, dann kann sich jedes Teammitglied entscheiden, ob er das Issue bearbeitet. Gründe für das direkte zuweisen können unteranderem sein, dass es eine entsprechende vorhergehende Absprache gab, dass der Architekt das Issue speziell einem Bereich zugehörig sieht bzw. eine spezielle Paarung erreichen möchte, oder aber auch, weil ein Issue schon zu lange in "Selected for Development" verweilt. Hat sich ein Teammitglied für ein Issue entschieden, trägt er sich als Bearbeiter ein und zieht es in auf "In Development".
	\item \textbf{In Development} \newline
	In dieser Spalte verweilen alle Issues, an denen gerade entwickelt wird. Wenn die Entwicklung an einem Issue abgeschlossen ist, zieht der Bearbeiter das Issue weiter auf "Needs Review".
	\item \textbf{Needs Review} \newline
	Hier verweilen alle Issues, deren Entwicklung abgeschlossen ist, aber noch nicht geprüft wurde, ob die Abnahmebedingungen erfüllt sind. Normalerweise sollte die Abnahme durch den Architekten erfolgen. Issues, die der Architekt bearbeitet hat, muss das Review von einem anderen Teammitglied gemacht werden. Ein Issue bei dem das Review durchgeführt wird, wird in die Spalte "In Review" verschoben.
	\item \textbf{In Review} \newline
	Hier sind alle Issues enthalten, die sich gerade im Review befinden. Sind alle Abnahmekriterien erfüllt, und sind durch die Bearbeitung des Issue keine neuen Probleme/Fehler hinzugekommen, wird es in die Spalte "Done" verschoben und das Issue geschlossen. Ist dies nicht der Fall, wird ein Entsprechender Kommentar mit einer möglichst detaillierten Beschreibung des Problems an das Issue angehängt, und es wieder auf "In Development" geschoben. 
	\item \textbf{Done} \newline
	Diese Spalte enthält alle abgeschlossenen Issues.
\end{itemize}

\subsection{Git}
Hier sind die Verhaltensweisen für die Nutzung von Git aufgeführt. Alles hier nicht aufgeführte kann von jedem Teammitglied nach eigenen ermessen gehandhabt werden.
\begin{itemize}
	\item \textbf{Branches} \newline
	Für jedes Issue wird ein Branch erstellt, außer es handelt sich um reine Dokumentation (im Ordner Docu). Ein Branchname folgt folgendem Muster: "\#<IssueNummer> <KurzerName>". Dadurch lässt sich 
	\item \textbf{Commits} \newline
	Commits folgen folgendem Namensschema: "\#<IssueNummer> <Beschreibung>".
	\item \textbf{Push und Pull} \newline
	Es sollte möglichst häufig gepusht werden, um einen eventuellen Datenverlust zu vermeiden. Beim Pull sollte mit "--rebase" gearbeitet werden, um die Historie möglichst sauber zu halten.
	\item \textbf{Merge und Pullrequest} \newline
	Bevor ein Issue auf "Needs Review" geschoben wird, ist der Master in den Branch zu mergen und ein Pullrequest (\href{https://github.com/Transport-Protocol/MBC-Ping-Pong/pulls}{https://github.com/Transport-Protocol/MBC-Ping-Pong/pulls}) zu erstellen. Derjenige der das Issue reviewt hat, merget den Branch dann mithilfe des Pullrequests in den Master und löscht ihn.
\end{itemize}

\section{Meilensteine}
In diesem Abschnitt werden die Meilensteine festgelegt. Hierbei wird beschrieben, was wan erreicht sein sollte.
\subsection{Projekt Aufsetzen}
\begin{itemize}
	\item \textbf{Beschreibung}\newline
	Die Grundlegenden für die Entwicklung notwendigen anfangs Infrastrukturen sind aufgesetzt.
	\item \textbf{Kriterien}
	\begin{itemize}
		\item \textbf{NodeJS-Server aufsetzen} \newline
		Der NodeJS Server ist aufgesetzt und stellt eine statische Website zur Verfügung
		\item \textbf{Docker} \newline
		Eine einheitliche Umgebung wird durch Docker und Docker-Compose ermöglicht.
	\end{itemize}
	\item \textbf{Beendet:} 25.11.2016
\end{itemize}

\subsection{Prototyp (Technik)}
\begin{itemize}
	\item \textbf{Beschreibung}\newline
	Um die identifizierten technischen Risiken schnellst möglich in den Griff zu bekommen, werden diese möglichst früh bearbeitet. In dem Prototyp (Technik) soll gezeigt werden, das die kritische Technik funktioniert. Dies wird anhand von kleine losgelösten Beispielen, die aber nahe der Zielarchitektur sind gezeigt.
	\item \textbf{Kriterien}
	\begin{itemize}
		\item \textbf{Darstellung} \newline
		Es wird gezeigt, das im Webbrowser eine Flüssige Darstellung möglich ist.
		\item \textbf{Kollisionserkennung} \newline
		Es wird gezeigt, dass eine Kollisionserkennung erreichbar ist.
		\item \textbf{Kommunikation mittels WebRTC} \newline
		Architektur bedingt ist die Nutzung von WebRTC unumgänglich. Es ist zu zeigen, dass eine Verbindung von mehreren Handys zum Darstellungsmedium möglich ist.
		\item \textbf{Steuerung} \newline
		Die Steuerung soll über den Touchscreen geschehen. Es ist zu zeigen, dass es möglich ist, die Position des Fingers auf dem Touchscreen im Browser auszulesen. 
		\item \textbf{Größen der Handys/Tabletts} \newline
		Unterschiedliche Handys und Tabletts haben verschiedene Größen und Formen. Somit ist ein Konzept zu erarbeiten, welches diesem Problem bei der Steuerung gerecht wird.
	\end{itemize}
	\item \textbf{Beendet:} 16.12.2016
\end{itemize}

\subsection{Release 1.0 (Zwei Spieler)}
\begin{itemize}
	\item \textbf{Beschreibung}\newline
	In diesem ersten Release ist eine Basisversion des Spieles implementiert. Es können zwei Spieler gegeneinander Spielen, indem sie ihre Schläger mit den Handys steuern. Gleichzeitig ist diese Version die minimal Version und enthält alle "Must"-Features.
	\item \textbf{Kriterien}
	\begin{itemize}
		\item \textbf{Schläger} \newline
		Für jeden Spieler existiert ein Schläger, der mit dem Handy Steuer
		\item \textbf{Ball} \newline
		Es gibt ein Ball, der sich über das Spielfeld bewegt. Kollidiert er mit einem Schläger oder der Wand, an der sich kein Schläger befindet, prallt er davon ab. Es gilt hierbei, dass der Einfallwinkel dem Ausfallwinkel entspricht. Zudem beschleunigt der Ball, wenn er mit einem Schläger kollidiert. Wenn der Ball mit der Wand hinter einem Schläger kollidiert, wir er in die Ausgangsposition versetzt und erhält die Ausgangsgeschwindigkeit.
		\item \textbf{Punkte} \newline
		Immer wenn der Ball mit der Wand hinter einem Schläger kollidiert, erhält der andere Spieler einen Punkt.
		\item \textbf{Spielende} \newline
		Das Spiel endet automatisch nach X Spielen (wobei gilt: $X \in  \mathbb{N} \land X \mod 2 = 1$ ). \textit{Das genaue X ist noch zu definieren.}
	\end{itemize}
	\item \textbf{Beendet:} 13.01.2017
\end{itemize}

\subsection{Release 1.X (Diverse Features)}
\begin{itemize}
	\item \textbf{Beschreibung}\newline
	Basierend auf der Version 1.0 wird das Spiel weiterentwickelt. Jedoch sind alle Features die hier bearbeitet werden können "Can"-Features. Daher kann es sein, dass das Release 1.X äquivalent zu dem Release 1.0 ist. Zudem sind alle hier genannten möglichen Features noch nicht näher spezifiziert und auf ihre Machbarkeit geprüft. Es gilt jedoch, dass je Umgesetztes Feature die Versionsnummer im Minorbereich um Eins steigt.
	\item \textbf{mögliche Kriterien}
	\begin{itemize}
		\item \textbf{N Spieler} \newline
		Mehr als 2 Spieler
		\item \textbf{Zusätzliche Hindernisse} \newline
		Auf dem Spielfeld sind zusätzliche Hindernisse.
		\item \textbf{Highscore-Liste} \newline
		Es wird eine Highscore-Liste geführt und angezeigt.
		\item \textbf{TBD} \newline
		To be discussed.
	\end{itemize}
	\item \textbf{Beendet:} 24.02.2017
\end{itemize}

\section{Highlevel View}
In diesem Abschnitt wird die Grob-/Gesamtarchitektur betrachtet. Hierbei wird nicht nur auf wesentliche Schnittstellen und Komponenten eingegangen. Zur engeren Auswahl standen zwei mögliche Ansätze. Es werden beide betrachtet, und erläutert warum der Ansatz 2 umgesetzt werden wird.
\subsection{Ansatz 1}
\begin{figure}[ht]
	\centering
	\includegraphics[width=0.9\textwidth]{architecture/highLevelAnsatz1.png}
	\caption{Highlevel Ansatz 1}
	\label{fig1}
\end{figure}

\subsection{Ansatz 2}
\begin{figure}[ht]
	\centering
	\includegraphics[width=0.9\textwidth]{architecture/highLevelAnsatz2.png}
	\caption{Highlevel Ansatz 2}
	\label{fig1}
\end{figure}





